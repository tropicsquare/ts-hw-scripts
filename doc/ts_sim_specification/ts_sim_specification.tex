%%%%%%%%%%%%%%%%%%%%%%%%%%%%%%%%%%%%%%%%%%%%%%%%%%%%%%%%%%%%%%%%%%%%%%%%%%%%%%%
%%%%%%%%%%%%%%%%%%%%%%%%%%%%%%%%%%%%%%%%%%%%%%%%%%%%%%%%%%%%%%%%%%%%%%%%%%%%%%%
% This is Specification for scripting system for digital simulations in
% Tropic Square. It specifies needed features for digital simulation scripting
% system.
% 
% This document uses Tropic Square LaTex library. Set file to this library in
% TROPICTEXLIBPATH variable (see below).
%
% LaTex class:
%	tropic_design_spec
%
%%%%%%%%%%%%%%%%%%%%%%%%%%%%%%%%%%%%%%%%%%%%%%%%%%%%%%%%%%%%%%%%%%%%%%%%%%%%%%%
%%%%%%%%%%%%%%%%%%%%%%%%%%%%%%%%%%%%%%%%%%%%%%%%%%%%%%%%%%%%%%%%%%%%%%%%%%%%%%%

% Specify Tropic Square document class
\documentclass{tropic_design_spec}


%%%%%%%%%%%%%%%%%%%%%%%%%%%%%%%%%%%%%%%%%%%%%%%%%%%%%%%%%%%%%%%%%%%%%%%%%%%%%%%
% Document properties and title page
%%%%%%%%%%%%%%%%%%%%%%%%%%%%%%%%%%%%%%%%%%%%%%%%%%%%%%%%%%%%%%%%%%%%%%%%%%%%%%%
\title{Design specification}
\author{Ondrej Ille, Tropic Square}
\date{July 2021}

% Start of document
\begin{document}

% Parameters Needed by Design spec class (must be inside document)
% Set these parameters according to your project.
\def \projectname {Tropic Square HW Simulation Scripting System}
\def \documentname {Design specification}
\def \versionnumber {0.3}

% Title page
\maketitle


%%%%%%%%%%%%%%%%%%%%%%%%%%%%%%%%%%%%%%%%%%%%%%%%%%%%%%%%%%%%%%%%%%%%%%%%%%%%%%%
% Document revisions
% We revision with GIT, however, it does not mean that major changes in the
% document should not be kept also with document!
% In general, when you increase document version number, add also entry to
% this table with revisions saying what changed!
%%%%%%%%%%%%%%%%%%%%%%%%%%%%%%%%%%%%%%%%%%%%%%%%%%%%%%%%%%%%%%%%%%%%%%%%%%%%%%%
\section*{Version history}

\begin{TropicRatioTable4Col}
	{0.1}			{0.2}				{0.2}			{0.5}
	{Version Tag 	& Date 				& Author		&	Description					}
     0.1            & 20.7.2021         & Ondrej Ille  	&	
     Initial version		\Ttlb

     0.2            & 10.8.2021         & Ondrej Ille  	&
     Adapt to first implementation. Mark covered for all requirements which are
     designed in first implementation. \Ttlb
     
     \versionnumber & 10.8.2021         & Ondrej Ille  	&
     Add specification of Requirement tracing.

\end{TropicRatioTable4Col}


%%%%%%%%%%%%%%%%%%%%%%%%%%%%%%%%%%%%%%%%%%%%%%%%%%%%%%%%%%%%%%%%%%%%%%%%%%%%%%%
% Table of contents
%%%%%%%%%%%%%%%%%%%%%%%%%%%%%%%%%%%%%%%%%%%%%%%%%%%%%%%%%%%%%%%%%%%%%%%%%%%%%%%
\pagebreak
\tableofcontents


%%%%%%%%%%%%%%%%%%%%%%%%%%%%%%%%%%%%%%%%%%%%%%%%%%%%%%%%%%%%%%%%%%%%%%%%%%%%%%%
%%%%%%%%%%%%%%%%%%%%%%%%%%%%%%%%%%%%%%%%%%%%%%%%%%%%%%%%%%%%%%%%%%%%%%%%%%%%%%%
% Document
%%%%%%%%%%%%%%%%%%%%%%%%%%%%%%%%%%%%%%%%%%%%%%%%%%%%%%%%%%%%%%%%%%%%%%%%%%%%%%%
%%%%%%%%%%%%%%%%%%%%%%%%%%%%%%%%%%%%%%%%%%%%%%%%%%%%%%%%%%%%%%%%%%%%%%%%%%%%%%%

%%%%%%%%%%%%%%%%%%%%%%%%%%%%%%%%%%%%%%%%%%%%%%%%%%%%%%%%%%%%%%%%%%%%%%%%%%%%%%%
% Introduction
%%%%%%%%%%%%%%%%%%%%%%%%%%%%%%%%%%%%%%%%%%%%%%%%%%%%%%%%%%%%%%%%%%%%%%%%%%%%%%%
\pagebreak
\section{Introduction}
This document specifies behavior of Digital simulation scripting system. This
system consists of set of scripts which can be used for following purposes:
\begin{itemize}
    \item{Define source list files, Define simulation tests}
    \item{Compile HDL code for simulation}
    \item{Simulate HW designs, run tests}
    \item{Check test results}
    \item{Run regressions}
    \item{Do requirement tracing}
    \item{Analyze code coverage}
\end{itemize}

At the moment, the scripting system is aimed towards VCS simulator, but the
structure shall be generic enough to add support of other simulators (either
Verilator for simple block level test-benches, GHDL, or other commercial simulators).


%%%%%%%%%%%%%%%%%%%%%%%%%%%%%%%%%%%%%%%%%%%%%%%%%%%%%%%%%%%%%%%%%%%%%%%%%%%%%%%
% Interface
%%%%%%%%%%%%%%%%%%%%%%%%%%%%%%%%%%%%%%%%%%%%%%%%%%%%%%%%%%%%%%%%%%%%%%%%%%%%%%%
\pagebreak
\section{Interface}

\subsection{Pre-requisities}

The scripting system relies on following pre-requisities:

\begin{itemize}
	\item{\textbf{TS_REPO_ROOT}\space}{Repository root variable}
\end{itemize}

At the moment it is assumed that \textbf{TS_REPO_ROOT} variable is set out-side
of scripting system. This is typically done by means of repository specific
\textit{setup_env} script.

\subsection{Scripts}
The scripting system contains following Python scripts:

\subsubsection*{ts_sim_compile.py}
Compiles simulation sources in current repository

\subsubsection*{ts_sim_run.py}
Runs the simulations/tests in current repository

\subsubsection*{ts_sim_check.py}
Checks test results

\subsubsection*{ts_sim_coverage.py}
Analyzes code coverage

\subsubsection*{ts_sim_regress.py}
Runs regression

\subsubsection*{ts_sim_req_trac.py}
Perform requirement tracing

\subsubsection*{ts_sim_config.yml}
YAML configuration file for whole system.


%%%%%%%%%%%%%%%%%%%%%%%%%%%%%%%%%%%%%%%%%%%%%%%%%%%%%%%%%%%%%%%%%%%%%%%%%%%%%%%
% Use-case
%%%%%%%%%%%%%%%%%%%%%%%%%%%%%%%%%%%%%%%%%%%%%%%%%%%%%%%%%%%%%%%%%%%%%%%%%%%%%%%
\pagebreak
\section{Use cases}

\subsection{Running digital simulation}
Following use-case demonstrate how can the user use this scripting system to
run digital simulations:
\begin{enumerate}
	\item{Check latest version of ts-hw-scripts available via \textit{ts_sw_cfg.py --list-sw}.}
	\item{Add this SW to repository specific config in \textit{ts_sw_cfg.yml}.
          Make sure that paths to simulator executables are in PATH variable}
    \item{Run \textit{source ./setup_env} from repository root folder.}
    \item{Run the scripts to compile/simulate HDL code}
\end{enumerate}


%%%%%%%%%%%%%%%%%%%%%%%%%%%%%%%%%%%%%%%%%%%%%%%%%%%%%%%%%%%%%%%%%%%%%%%%%%%%%%%%%%%%%%%%%
% Common requirements
%%%%%%%%%%%%%%%%%%%%%%%%%%%%%%%%%%%%%%%%%%%%%%%%%%%%%%%%%%%%%%%%%%%%%%%%%%%%%%%%%%%%%%%%%
\pagebreak

\section{Common requirements}

\ReqStart{TS_SCMN}

%%%%%%%%%%%%%%%%%%%%%%%%%%%%%%%%%%%%%%%%%%%%%%%%%%%%%%%%%%%%%%%%%%%%%%%%%%%%%%%%%%%%%%%%%

\ReqItem{1}
    {If there is an option in simulation config file which can be overiden by a command
     line switch of a script, such switch shall always have priority (if it is set).}

%%%%%%%%%%%%%%%%%%%%%%%%%%%%%%%%%%%%%%%%%%%%%%%%%%%%%%%%%%%%%%%%%%%%%%%%%%%%%%%%%%%%%%%%%

\ReqItem{1}
    {If there is an option in simulation config file which can be overriden by a command
     line switch of a script, then the name of this option shall be equal to switch name
     with the exception of word separator. (E.g. switch '--clear-logs' shall correspond
     to keyword 'clear_logs'.}

%%%%%%%%%%%%%%%%%%%%%%%%%%%%%%%%%%%%%%%%%%%%%%%%%%%%%%%%%%%%%%%%%%%%%%%%%%%%%%%%%%%%%%%%%

\ReqItem{1}
    {Word separator for command line switches shall be "-", word separator for config
     file, source/test list file keywords shall be "_".}

%%%%%%%%%%%%%%%%%%%%%%%%%%%%%%%%%%%%%%%%%%%%%%%%%%%%%%%%%%%%%%%%%%%%%%%%%%%%%%%%%%%%%%%%%

\ReqItem{1}
    {All script shall have "--help", similar to standard Unix help command. Help shall
     show all command line options.}

%%%%%%%%%%%%%%%%%%%%%%%%%%%%%%%%%%%%%%%%%%%%%%%%%%%%%%%%%%%%%%%%%%%%%%%%%%%%%%%%%%%%%%%%%

\ReqItem{1}{All script shall have following "verbosity" options:}
\ReqSubStart
    \ReqSubItem{1}{Regular verbosity - Shows only most important steps executed by script.}
    \ReqSubItem{1}{Verbose - Shows nearly all steps executed by a script.}
    \ReqSubItem{1}{Very verbose - Shows all steps executed by a script}
\ReqSubEnd

%%%%%%%%%%%%%%%%%%%%%%%%%%%%%%%%%%%%%%%%%%%%%%%%%%%%%%%%%%%%%%%%%%%%%%%%%%%%%%%%%%%%%%%%%

\ReqItem{1}{If applicable, verbosity options shall be passed to underlying simulator,
            compiler.}

%%%%%%%%%%%%%%%%%%%%%%%%%%%%%%%%%%%%%%%%%%%%%%%%%%%%%%%%%%%%%%%%%%%%%%%%%%%%%%%%%%%%%%%%%

\ReqItem{1}{All scripts shall use colorized outputs to command line to distinguish
            importance of various messages (Error - Red, Warning - Yellow/Orange, Info -
            Purple). The reason behind this is to ease debugging/locating of problems from
            the script output. It shall be possible to disable colorized outputs (in case
            script is executed in environment without ANSI coloring).}

%%%%%%%%%%%%%%%%%%%%%%%%%%%%%%%%%%%%%%%%%%%%%%%%%%%%%%%%%%%%%%%%%%%%%%%%%%%%%%%%%%%%%%%%%

\ReqEnd


%%%%%%%%%%%%%%%%%%%%%%%%%%%%%%%%%%%%%%%%%%%%%%%%%%%%%%%%%%%%%%%%%%%%%%%%%%%%%%%%%%%%%%%%%
% TS_SIM_CONFIG
%%%%%%%%%%%%%%%%%%%%%%%%%%%%%%%%%%%%%%%%%%%%%%%%%%%%%%%%%%%%%%%%%%%%%%%%%%%%%%%%%%%%%%%%%
\pagebreak

\section{ts_sim_config.yml}

\ReqStart{TS_SCFG}

%%%%%%%%%%%%%%%%%%%%%%%%%%%%%%%%%%%%%%%%%%%%%%%%%%%%%%%%%%%%%%%%%%%%%%%%%%%%%%%%%%%%%%%%%

\ReqItem{1}
    {Scripting system shall have configuration available in YAML config file                       
     \textbf{ts_sim_config.yml} located in TS_REPO_ROOT/\textbf{sim} directory}

%%%%%%%%%%%%%%%%%%%%%%%%%%%%%%%%%%%%%%%%%%%%%%%%%%%%%%%%%%%%%%%%%%%%%%%%%%%%%%%%%%%%%%%%%

\ReqItem{1}{Configuration file shall contain:}
\begin{itemize}
    \item{Current simulation tool to be used. Right now only VCS shall be supported.}
    \item{List of design "targets". Design "target" is similar to makefile target, and
          it can be used to distinguish between RTL, FPGA and gate level simulation.}	
    \item{VHDL/Verilog language versions}
    \item{Tool specific compile, elaborate and simulate options. It shall be possible
          to specify common options and also options separate for each tool.}
    \item{Enable/Disable of compilation with coverage collection.}
    \item{Ability to enable/disable "debug mode" for compiler underlying the simulator.
          It shall be possible to specify usage of these optimizations by an option.}
	\item{Global macros defined in all Verilog/System Verilog files}
	\item{List of regular expressions which cause that line from log file is classified
	      as Error when it matches this regular expression.}
	\item{List of regular expressions which cause that line from log file is classified
	      as Warning when it matches this regular expression.}
	\item{Simulation "check severity" with possible values: Error, Warning. With "Error"
	      severity, if there is a line in simulation log file which is classified as
	      Error, test fails. With "Warning" severity, if there is a line in simulation
	      log file which is classified as Error or Warning, test fails.}
    \item{Regular expression patterns which allow to turn on/off ignoring of lines which
          are classified as Error/Warnings. If the line is ignored, it does not cause
          test failure.}
    \item{Time resolution for simulation. The system shall abstract away differences
          between VHDL and Verilog/System Verilog resolution and single resolution shall
          configure both VHDL and Verilog/System Verilog designs. If Verilog/System
          Verilog designs have file specific resolution, the system shall not override
          it.}
    \item{Simulation verbosity. Mapping of simulation verbosity to framework specific
          verbosity (e.g. UVM verbosity) shall be possible via configuration file.}
\end{itemize}

%%%%%%%%%%%%%%%%%%%%%%%%%%%%%%%%%%%%%%%%%%%%%%%%%%%%%%%%%%%%%%%%%%%%%%%%%%%%%%%%%%%%%%%%%

\ReqItem{1}{Each design target shall have following configurable:}
\begin{itemize}
	\item{Target name}
	\item{List of source list files. See "Source list files" section.}
	\item{Target specific compilation options}
	\item{Target specific simulation options}
	\item{Simulation top module/entity}
	\item{Settings of top level generics/parameters}
\end{itemize}

%%%%%%%%%%%%%%%%%%%%%%%%%%%%%%%%%%%%%%%%%%%%%%%%%%%%%%%%%%%%%%%%%%%%%%%%%%%%%%%%%%%%%%%%%

\ReqItem{1}
    {Path to files (e.g. path to source/test list files) in simulation config file shall
     be treated as relative to \textbf{TS_REPO_ROOT} variable if they are relative. If
     the paths are absolute, then they shall be treated as absolute.}
            
%%%%%%%%%%%%%%%%%%%%%%%%%%%%%%%%%%%%%%%%%%%%%%%%%%%%%%%%%%%%%%%%%%%%%%%%%%%%%%%%%%%%%%%%%

\ReqEnd


%%%%%%%%%%%%%%%%%%%%%%%%%%%%%%%%%%%%%%%%%%%%%%%%%%%%%%%%%%%%%%%%%%%%%%%%%%%%%%%%%%%%%%%%%
% Source list files
%%%%%%%%%%%%%%%%%%%%%%%%%%%%%%%%%%%%%%%%%%%%%%%%%%%%%%%%%%%%%%%%%%%%%%%%%%%%%%%%%%%%%%%%%

\pagebreak
\section{Source list files}

\ReqStart{TS_SLST}

%%%%%%%%%%%%%%%%%%%%%%%%%%%%%%%%%%%%%%%%%%%%%%%%%%%%%%%%%%%%%%%%%%%%%%%%%%%%%%%%%%%%%%%%%

\ReqItem{1}
    {Repository RTL and TB source codes shall be listed in YAML source list files. Each
     source file shall be explicitly listed, auto-discovery of source files shall be
     avoided}.

%%%%%%%%%%%%%%%%%%%%%%%%%%%%%%%%%%%%%%%%%%%%%%%%%%%%%%%%%%%%%%%%%%%%%%%%%%%%%%%%%%%%%%%%%

\ReqItem{1}
    {Paths to individual source files within a source list file shall be interpreted as
     relative to current list file, if they are absolute. If paths to a file is absolute,
     it shall be treated as absolute.}

%%%%%%%%%%%%%%%%%%%%%%%%%%%%%%%%%%%%%%%%%%%%%%%%%%%%%%%%%%%%%%%%%%%%%%%%%%%%%%%%%%%%%%%%%

\ReqItem{1}
    {Source list files shall support VHDL, Verilog, System Verilog and/or YAML files}.

%%%%%%%%%%%%%%%%%%%%%%%%%%%%%%%%%%%%%%%%%%%%%%%%%%%%%%%%%%%%%%%%%%%%%%%%%%%%%%%%%%%%%%%%%

\ReqItem{1}
    {Type of file shall be deduced from file extension by all scripts. Scripting system
     shall treat each file based on its extension (e.g. calling vlogan for Verilog files,
     and vhdlan VHDL files). Following extensions shall be supported:}
\ReqSubStart
	\ReqSubItem{1}{".vhd" - VHDL}
	\ReqSubItem{1}{".sv"  - System Verilog}
	\ReqSubItem{1}{".v"	  - Verilog}
	\ReqSubItem{1}{".yml" - YAML - Indicates nested list file}
\ReqSubEnd

%%%%%%%%%%%%%%%%%%%%%%%%%%%%%%%%%%%%%%%%%%%%%%%%%%%%%%%%%%%%%%%%%%%%%%%%%%%%%%%%%%%%%%%%%

\ReqItem{1}
    {VHDL, Verilog, SystemVerilog files from source list file shall be compiled by a      
     simulator which is configured in simulator config file.}

%%%%%%%%%%%%%%%%%%%%%%%%%%%%%%%%%%%%%%%%%%%%%%%%%%%%%%%%%%%%%%%%%%%%%%%%%%%%%%%%%%%%%%%%%

\ReqItem{1}
    {YAML file shall be interpreted as nested list file and each script analyzing such
     list file shall recursively analyze this nested source list file. The script shall
     implement a mechanism to limit depth of nesting of source list files.}

%%%%%%%%%%%%%%%%%%%%%%%%%%%%%%%%%%%%%%%%%%%%%%%%%%%%%%%%%%%%%%%%%%%%%%%%%%%%%%%%%%%%%%%%%

\ReqItem{1}
    {It shall be possible to define library into which source file will be compiled for
     each file, as well as for all files within a list file.}

%%%%%%%%%%%%%%%%%%%%%%%%%%%%%%%%%%%%%%%%%%%%%%%%%%%%%%%%%%%%%%%%%%%%%%%%%%%%%%%%%%%%%%%%%

\ReqItem{1}
    {It shall be possible to define custom macro definition for each source file, as well
     as for all files within a list file.}

%%%%%%%%%%%%%%%%%%%%%%%%%%%%%%%%%%%%%%%%%%%%%%%%%%%%%%%%%%%%%%%%%%%%%%%%%%%%%%%%%%%%%%%%%

\ReqItem{1}
    {Per-file tool specific switches shall be avoided in source list files (apart from
     macro define). The reason is that whole repository shall be compilable with the same
     configuration settings. It is not appropriate to e.g. compile RTL with one version
     of System Verilog standard and TB with another version of System Verilog standard.}

%%%%%%%%%%%%%%%%%%%%%%%%%%%%%%%%%%%%%%%%%%%%%%%%%%%%%%%%%%%%%%%%%%%%%%%%%%%%%%%%%%%%%%%%%

\ReqEnd


%%%%%%%%%%%%%%%%%%%%%%%%%%%%%%%%%%%%%%%%%%%%%%%%%%%%%%%%%%%%%%%%%%%%%%%%%%%%%%%%%%%%%%%%%
% Test list files
%%%%%%%%%%%%%%%%%%%%%%%%%%%%%%%%%%%%%%%%%%%%%%%%%%%%%%%%%%%%%%%%%%%%%%%%%%%%%%%%%%%%%%%%%

\pagebreak
\section{Test list files}

\ReqStart{TS_TLST}

%%%%%%%%%%%%%%%%%%%%%%%%%%%%%%%%%%%%%%%%%%%%%%%%%%%%%%%%%%%%%%%%%%%%%%%%%%%%%%%%%%%%%%%%%

\ReqItem{1}
    {It shall be possible to define available tests in current repository via test list
     file. Test list file shall be YAML file.}
     
%%%%%%%%%%%%%%%%%%%%%%%%%%%%%%%%%%%%%%%%%%%%%%%%%%%%%%%%%%%%%%%%%%%%%%%%%%%%%%%%%%%%%%%%%

\ReqItem{1}
    {It shall be possible to group tests into groups in test list file. Each group of
     tests shall have its name.}

%%%%%%%%%%%%%%%%%%%%%%%%%%%%%%%%%%%%%%%%%%%%%%%%%%%%%%%%%%%%%%%%%%%%%%%%%%%%%%%%%%%%%%%%%

\ReqItem{1}
    {Each test shall have following configurable:}
\ReqSubStart
    \ReqSubItem{1}{Test name}
    \ReqSubItem{1}{Elaboration and Simulation options which are test specific.}
    \ReqSubItem{1}{Pre-test and Post-test hooks.}
    \ReqSubItem{1}{Top level generics which are test specific.}
\ReqSubEnd

%%%%%%%%%%%%%%%%%%%%%%%%%%%%%%%%%%%%%%%%%%%%%%%%%%%%%%%%%%%%%%%%%%%%%%%%%%%%%%%%%%%%%%%%%

\ReqItem{1}
    {It shall be possible to load sub-test list files. This allows nesting test list
     files. All scripts shall recursively search this sub-list file for available tests.
     It shall be possible to merge tests from sub-list files into a flat list of tests,
     as well as placing them into a test group.}

%%%%%%%%%%%%%%%%%%%%%%%%%%%%%%%%%%%%%%%%%%%%%%%%%%%%%%%%%%%%%%%%%%%%%%%%%%%%%%%%%%%%%%%%%

\ReqEnd


%%%%%%%%%%%%%%%%%%%%%%%%%%%%%%%%%%%%%%%%%%%%%%%%%%%%%%%%%%%%%%%%%%%%%%%%%%%%%%%%%%%%%%%%%
% TS_SIM_COMPILE
%%%%%%%%%%%%%%%%%%%%%%%%%%%%%%%%%%%%%%%%%%%%%%%%%%%%%%%%%%%%%%%%%%%%%%%%%%%%%%%%%%%%%%%%%
\pagebreak

\section{ts_sim_compile.py}

\ReqStart{TS_SCOMP}

%%%%%%%%%%%%%%%%%%%%%%%%%%%%%%%%%%%%%%%%%%%%%%%%%%%%%%%%%%%%%%%%%%%%%%%%%%%%%%%%%%%%%%%%%

\ReqItem{1}
    {Scripting system shall implement \textbf{ts_sim_compile.py} script. The main
     purpose of the script is to compile all sources for simulation in current 	
     repository. This script is further in this section referred to as "script".}

%%%%%%%%%%%%%%%%%%%%%%%%%%%%%%%%%%%%%%%%%%%%%%%%%%%%%%%%%%%%%%%%%%%%%%%%%%%%%%%%%%%%%%%%%

\ReqItem{1}
    {Script shall by default compile all sources to \textbf{TS_REPO_ROOT/sim/build}
     folder}

\ReqSubStart
    \ReqSubItem{1}
        {It shall be possible to override the directory where the code is 
         compiled either in simulation_config file, or via command line switch.}
\ReqSubEnd

%%%%%%%%%%%%%%%%%%%%%%%%%%%%%%%%%%%%%%%%%%%%%%%%%%%%%%%%%%%%%%%%%%%%%%%%%%%%%%%%%%%%%%%%%

\ReqItem{1}
    {Script shall perform "compile" step of commnly known three-step simulation flow.}

%%%%%%%%%%%%%%%%%%%%%%%%%%%%%%%%%%%%%%%%%%%%%%%%%%%%%%%%%%%%%%%%%%%%%%%%%%%%%%%%%%%%%%%%%

\ReqItem{1}
    {Sources for compilation shall be given by reference to a single or more lists file 
     in \textbf{ts_sim_config.yml} config file.}%
		
%%%%%%%%%%%%%%%%%%%%%%%%%%%%%%%%%%%%%%%%%%%%%%%%%%%%%%%%%%%%%%%%%%%%%%%%%%%%%%%%%%%%%%%%%

\ReqItem{0}
    {If underlying simulator does not support particular file type (.e.g. VHDL in case of
     Verilator), script shall indicate this and throw an error.}%

%%%%%%%%%%%%%%%%%%%%%%%%%%%%%%%%%%%%%%%%%%%%%%%%%%%%%%%%%%%%%%%%%%%%%%%%%%%%%%%%%%%%%%%%%

\ReqItem{1}
    {Script shall be configurable by a \textbf{ts_sim_config.yml} config file.}%
\ReqSubStart
    \ReqSubItem{1}
        {Script shall take \textbf{sim/ts_sim_config.yml} file as config file.}
    \ReqSubItem{1}
        {It shall be possible to override used simulation config file.}
\ReqSubEnd

%%%%%%%%%%%%%%%%%%%%%%%%%%%%%%%%%%%%%%%%%%%%%%%%%%%%%%%%%%%%%%%%%%%%%%%%%%%%%%%%%%%%%%%%%

\ReqItem{1}
    {Script shall have an option to specify for which target from config file you are
     compiling, as well as option to list all available targets, and list all source files
     within a target.}

%%%%%%%%%%%%%%%%%%%%%%%%%%%%%%%%%%%%%%%%%%%%%%%%%%%%%%%%%%%%%%%%%%%%%%%%%%%%%%%%%%%%%%%%%

\ReqItem{1}
    {Script shall have an option to clear all compiled files (force recompile).}

%%%%%%%%%%%%%%%%%%%%%%%%%%%%%%%%%%%%%%%%%%%%%%%%%%%%%%%%%%%%%%%%%%%%%%%%%%%%%%%%%%%%%%%%%

\ReqItem{1}
    {Script shall store compile logs to \textbf{sim/comp_logs} folder.}
			
%%%%%%%%%%%%%%%%%%%%%%%%%%%%%%%%%%%%%%%%%%%%%%%%%%%%%%%%%%%%%%%%%%%%%%%%%%%%%%%%%%%%%%%%%

\ReqItem{1}
    {Script shall terminate and clearly indicate error as soon as compilation of any file
     fails.}

%%%%%%%%%%%%%%%%%%%%%%%%%%%%%%%%%%%%%%%%%%%%%%%%%%%%%%%%%%%%%%%%%%%%%%%%%%%%%%%%%%%%%%%%%

\ReqItem{1}
    {Compilation log file shall contain target name and timestamp when compilation was
     performed. When for multiple targets, log files will not be over-written.}

%%%%%%%%%%%%%%%%%%%%%%%%%%%%%%%%%%%%%%%%%%%%%%%%%%%%%%%%%%%%%%%%%%%%%%%%%%%%%%%%%%%%%%%%%

\ReqItem{1}
    {Script shall call pre-compile and post-compile hooks as implemented in 
     \textbf{ts_sim_config.yml} file. Pre-compile hook shall execute before compilation.
     Post-compile hook shall execute after simulation. Each hook (if set), shall be a
     path to separate Python or Bash script with commands to be executed. Typical
     use-case is compilation of e.g. SW for simulation of CPU core ia external
     compilation system by running "make" command.}

%%%%%%%%%%%%%%%%%%%%%%%%%%%%%%%%%%%%%%%%%%%%%%%%%%%%%%%%%%%%%%%%%%%%%%%%%%%%%%%%%%%%%%%%%

\ReqEnd


%%%%%%%%%%%%%%%%%%%%%%%%%%%%%%%%%%%%%%%%%%%%%%%%%%%%%%%%%%%%%%%%%%%%%%%%%%%%%%%%%%%%%%%%%
% TS_SIM_RUN
%%%%%%%%%%%%%%%%%%%%%%%%%%%%%%%%%%%%%%%%%%%%%%%%%%%%%%%%%%%%%%%%%%%%%%%%%%%%%%%%%%%%%%%%%
\pagebreak

\section{ts_sim_run.py}

\ReqStart{TS_SRUN}

%%%%%%%%%%%%%%%%%%%%%%%%%%%%%%%%%%%%%%%%%%%%%%%%%%%%%%%%%%%%%%%%%%%%%%%%%%%%%%%%%%%%%%%%%

\ReqItem{1}
    {Scripting system shall implement \textbf{ts_sim_run.py} script. This script shall be
     used to launch simulations in current repository. \textbf{ts_sim_run.py} is further
     in this section only referred to as "script".}

%%%%%%%%%%%%%%%%%%%%%%%%%%%%%%%%%%%%%%%%%%%%%%%%%%%%%%%%%%%%%%%%%%%%%%%%%%%%%%%%%%%%%%%%%

\ReqItem{1}
    {Script shall rely on TS_REPO_ROOT environment variable, and \textbf{ts_sim_config.yml}
     configuration file.}

%%%%%%%%%%%%%%%%%%%%%%%%%%%%%%%%%%%%%%%%%%%%%%%%%%%%%%%%%%%%%%%%%%%%%%%%%%%%%%%%%%%%%%%%%

\ReqItem{1}
    {Script depends on \textbf{ts_sim_compile.py} script. Therefore, the simulation flow
     shall be:
    \begin{enumerate}			
        \item{"compile" (\textbf{ts_sim_compile.py})}
        \item{"simulate" (\textbf{ts_sim_run.py})}
    \end{enumerate}
     It is users responsibility to call \textbf{ts_sim_compile.py} script first.}

%%%%%%%%%%%%%%%%%%%%%%%%%%%%%%%%%%%%%%%%%%%%%%%%%%%%%%%%%%%%%%%%%%%%%%%%%%%%%%%%%%%%%%%%%

\ReqItem{1}
    {Script shall perform "elaborate" and "simulate" parts of the three-step simulation
     flow automatically as single step.}.	

%%%%%%%%%%%%%%%%%%%%%%%%%%%%%%%%%%%%%%%%%%%%%%%%%%%%%%%%%%%%%%%%%%%%%%%%%%%%%%%%%%%%%%%%%

\ReqItem{1}
    {It shall be possible to only instruct the script to perform elaboration, and not
     launch simulation by a command line switch}.

%%%%%%%%%%%%%%%%%%%%%%%%%%%%%%%%%%%%%%%%%%%%%%%%%%%%%%%%%%%%%%%%%%%%%%%%%%%%%%%%%%%%%%%%%

\ReqItem{1}
    {By default, script shall run simulation in batch mode. If specified by a switch,
     script shall launch simulation in GUI mode.}.
\ReqSubStart
    \ReqSubItem{1}{VCS simulator shall support launching DVE when running in GUI mode.}
\ReqSubEnd

%%%%%%%%%%%%%%%%%%%%%%%%%%%%%%%%%%%%%%%%%%%%%%%%%%%%%%%%%%%%%%%%%%%%%%%%%%%%%%%%%%%%%%%%%

\ReqItem{1}
    {To perform simulation, user shall put design target equal to compilation design
     target as a command line argument. It is users responsibility that compilation was
     executed for the same target as simulation.}.

%%%%%%%%%%%%%%%%%%%%%%%%%%%%%%%%%%%%%%%%%%%%%%%%%%%%%%%%%%%%%%%%%%%%%%%%%%%%%%%%%%%%%%%%%

\ReqItem{1}
    {Script shall have a switch which will enable recording of full design hierarchy into
     wave files}.

%%%%%%%%%%%%%%%%%%%%%%%%%%%%%%%%%%%%%%%%%%%%%%%%%%%%%%%%%%%%%%%%%%%%%%%%%%%%%%%%%%%%%%%%%

\ReqItem{1}
    {It shall be possible to specify test name when running the script. The script shall
     check whether test matches any available test name from any of test list files. If
     not, the script shall indicate that there is no such test available.} 

%%%%%%%%%%%%%%%%%%%%%%%%%%%%%%%%%%%%%%%%%%%%%%%%%%%%%%%%%%%%%%%%%%%%%%%%%%%%%%%%%%%%%%%%%

\ReqItem{1}
    {It shall be possible to use wild-cards to specify test names.}

%%%%%%%%%%%%%%%%%%%%%%%%%%%%%%%%%%%%%%%%%%%%%%%%%%%%%%%%%%%%%%%%%%%%%%%%%%%%%%%%%%%%%%%%%

\ReqItem{1}
    {Location of root test list file shall be configurable via \textbf{ts_sim_config.yml}
    simulation config file.}

%%%%%%%%%%%%%%%%%%%%%%%%%%%%%%%%%%%%%%%%%%%%%%%%%%%%%%%%%%%%%%%%%%%%%%%%%%%%%%%%%%%%%%%%%

\ReqItem{1}
    {When running a test as an argument of a script, tests from groups shall be accessible
     by using group name as prefix to test name. This avoids naming conflicts between
     tests in groups.}

%%%%%%%%%%%%%%%%%%%%%%%%%%%%%%%%%%%%%%%%%%%%%%%%%%%%%%%%%%%%%%%%%%%%%%%%%%%%%%%%%%%%%%%%%

\ReqItem{1}
    {It shall be possible to over-ride seed for randomization when starting the script.
     This allows debugging of fails from regressions. If seed is not specified by 
     command line switch, or simulation config file, random seed shall be generated!}

%%%%%%%%%%%%%%%%%%%%%%%%%%%%%%%%%%%%%%%%%%%%%%%%%%%%%%%%%%%%%%%%%%%%%%%%%%%%%%%%%%%%%%%%%

\ReqItem{1}
    {It shall be possible to run a test multiple times in a loop by a command line switch.}

%%%%%%%%%%%%%%%%%%%%%%%%%%%%%%%%%%%%%%%%%%%%%%%%%%%%%%%%%%%%%%%%%%%%%%%%%%%%%%%%%%%%%%%%%

\ReqItem{1}
    {It shall be possible to specify multiple tests at single call of the script. Each
     test shall be executed once, or multiple times if specified by a switch from previous
     requirement.}

%%%%%%%%%%%%%%%%%%%%%%%%%%%%%%%%%%%%%%%%%%%%%%%%%%%%%%%%%%%%%%%%%%%%%%%%%%%%%%%%%%%%%%%%%

\ReqItem{1}
    {When multiple tests are run (either repetition of single test, or different tests),
     script shall execute these strictly synchronously, therefore consuming one license
     at a time. For parallel execution refer to \textbf{ts_sim_regress.py} script.}

%%%%%%%%%%%%%%%%%%%%%%%%%%%%%%%%%%%%%%%%%%%%%%%%%%%%%%%%%%%%%%%%%%%%%%%%%%%%%%%%%%%%%%%%%

\ReqItem{1}
    {Test log files from simulation shall be stored in \textbf{TS_REPO_ROOT/sim_logs/}.
     Name of each log file shall contain timestamp, test name, design target and seed.
     Script shall append simulator exit code and duration of simulation to each log file.}

%%%%%%%%%%%%%%%%%%%%%%%%%%%%%%%%%%%%%%%%%%%%%%%%%%%%%%%%%%%%%%%%%%%%%%%%%%%%%%%%%%%%%%%%%

\ReqItem{1}
    {Script shall run elaboration and simulation (create simulation binary) into a separate
     folder for each test. The folder shall be located under "build" directory in which
     code was compiled.}

%%%%%%%%%%%%%%%%%%%%%%%%%%%%%%%%%%%%%%%%%%%%%%%%%%%%%%%%%%%%%%%%%%%%%%%%%%%%%%%%%%%%%%%%%

\ReqItem{1}
    {When simulation ends, script shall invoke \textbf{ts_sim_check.py} script (see
     separate section for its specification). Script shall pass name of the simulation
     log file(s) to the \textbf{ts_sim_check.py}. Script shall return a return value passed
     from this sub-script.}

%%%%%%%%%%%%%%%%%%%%%%%%%%%%%%%%%%%%%%%%%%%%%%%%%%%%%%%%%%%%%%%%%%%%%%%%%%%%%%%%%%%%%%%%%

\ReqItem{1}
    {It shall be possible to configure if scripts shall proceed with simulating next test
     if \textbf{ts_sim_check.py} marks the test as "failed" (\textbf{ts_sim_check.py}
     returns non-zero value). By default, the simulation shall run all the tests.}

%%%%%%%%%%%%%%%%%%%%%%%%%%%%%%%%%%%%%%%%%%%%%%%%%%%%%%%%%%%%%%%%%%%%%%%%%%%%%%%%%%%%%%%%%

\ReqItem{1}
    {If there were multiple tests ran, the script shall print summary of all the test
     results as given by \textbf{ts_sim_check.py} for each test. This summary shall be
     printed after all tests were ran.}

%%%%%%%%%%%%%%%%%%%%%%%%%%%%%%%%%%%%%%%%%%%%%%%%%%%%%%%%%%%%%%%%%%%%%%%%%%%%%%%%%%%%%%%%%

\ReqItem{1}
    {It shall be possible to disable call of the \textbf{ts_sim_check.py} script at the
     end of \textbf{ts_sim_run.py} execution by a command line switch. However, by default,
     the script shall call \textbf{ts_sim_check.py}.}

%%%%%%%%%%%%%%%%%%%%%%%%%%%%%%%%%%%%%%%%%%%%%%%%%%%%%%%%%%%%%%%%%%%%%%%%%%%%%%%%%%%%%%%%%

\ReqItem{1}
    {It shall be possible to specify following hooks for the script:

\begin{itemize}
    \item{Pre-run            - Executes before start of first test.}
    \item{Pre-test           - Executes before start of each test - common to all tests}
    \item{Pre-test-specific  - Executes before start of each test - test specific}
    \item{Post-test-specific - Executes after finish of each test - test specific}
    \item{Post-test          - Executes after finish of each test - common to all tests}
    \item{Post-run           - Executes after finish of all test, before run of
                               \textbf{ts_sim_check.py}}
    \item{Post-check         - Executes after finish of \textbf{ts_sim_check.py} script
                               executed at the end of test-run.}
\end{itemize}

    Each hook shall be a path to Python/Bash script, and it shall be specified via
    \textbf{ts_sim_config.yml} configuration file, or could be test specific (specified
    in test list file).
}.

%%%%%%%%%%%%%%%%%%%%%%%%%%%%%%%%%%%%%%%%%%%%%%%%%%%%%%%%%%%%%%%%%%%%%%%%%%%%%%%%%%%%%%%%%

\ReqItem{1}
    {It shall be possible to pass additional elaboration and simulation options by a
     specific command line switch.}

%%%%%%%%%%%%%%%%%%%%%%%%%%%%%%%%%%%%%%%%%%%%%%%%%%%%%%%%%%%%%%%%%%%%%%%%%%%%%%%%%%%%%%%%%

\ReqEnd



%%%%%%%%%%%%%%%%%%%%%%%%%%%%%%%%%%%%%%%%%%%%%%%%%%%%%%%%%%%%%%%%%%%%%%%%%%%%%%%%%%%%%%%%%
% TS_SIM_CHECK
%%%%%%%%%%%%%%%%%%%%%%%%%%%%%%%%%%%%%%%%%%%%%%%%%%%%%%%%%%%%%%%%%%%%%%%%%%%%%%%%%%%%%%%%%
\pagebreak

\section{ts_sim_check.py}

\ReqStart{TS_SCHK}

%%%%%%%%%%%%%%%%%%%%%%%%%%%%%%%%%%%%%%%%%%%%%%%%%%%%%%%%%%%%%%%%%%%%%%%%%%%%%%%%%%%%%%%%%

\ReqItem{1}
    {Scripting system shall implement \textbf{ts_sim_check.py} script whose main purpose
     is to determine if simulation passed or failed based on test log file. Further, this
     section refers to it only as "script".}

%%%%%%%%%%%%%%%%%%%%%%%%%%%%%%%%%%%%%%%%%%%%%%%%%%%%%%%%%%%%%%%%%%%%%%%%%%%%%%%%%%%%%%%%%

\ReqItem{1}
    {Script shall be invoked by \textbf{ts_sim_run.py} script, and it shall take name of
     the log file as an argument. It shall be possible to provide multiple log files
     at once.}

%%%%%%%%%%%%%%%%%%%%%%%%%%%%%%%%%%%%%%%%%%%%%%%%%%%%%%%%%%%%%%%%%%%%%%%%%%%%%%%%%%%%%%%%%

\ReqItem{1}
    {It shall be possible to invoke the script separately. In such case log files provided
     shall be provided via command line switch. If the log file paths are relative, they
     shall be interpreted as relative to directory from which the script is called.}

%%%%%%%%%%%%%%%%%%%%%%%%%%%%%%%%%%%%%%%%%%%%%%%%%%%%%%%%%%%%%%%%%%%%%%%%%%%%%%%%%%%%%%%%%

\ReqItem{1}
    {Script shall determine whether the test(s) has passed or failed. Number of failed
     tests shall be returned as script exit code.}

%%%%%%%%%%%%%%%%%%%%%%%%%%%%%%%%%%%%%%%%%%%%%%%%%%%%%%%%%%%%%%%%%%%%%%%%%%%%%%%%%%%%%%%%%

\ReqItem{1}
    {If the simulation exited with non-zero exit code, test shall be marked as failed.}

%%%%%%%%%%%%%%%%%%%%%%%%%%%%%%%%%%%%%%%%%%%%%%%%%%%%%%%%%%%%%%%%%%%%%%%%%%%%%%%%%%%%%%%%%

\ReqItem{1}
    {The script shall parse input log file(s) and look for specific patterns which
     determine if the test has failed. These patterns shall be regular expressions,
     and script shall search each line of the provided log file(s) for their occurrence.
     Following logs shall cause the test to fail:
\begin{itemize}
	\item{Any line within log file classified as "Error"}
	\item{Any line within log file classified as "Warning" shall cause test to fail if it
		  is configured in \textbf{ts_sim_config.yml} config file.}
\end{itemize}

    Note that format of these messages is simulator specific! Further, it shall be
    possible for user to extend these!

    The reason why we shall not rely purely on "UVM_ERROR" reporting mechanism is, that
    analog models, memory models delivered by third parties might not use UVM, but they
    will surely use "error" severity logs to throw errors. Therefore it is important to
    create a system which decides "what passed" and "what failed" not purely on number
    of UVM_ERRORs, but is rather super-set of UVM reporting mechanisms.
}


%%%%%%%%%%%%%%%%%%%%%%%%%%%%%%%%%%%%%%%%%%%%%%%%%%%%%%%%%%%%%%%%%%%%%%%%%%%%%%%%%%%%%%%%%

\ReqItem{1}
    {Script shall count number warnings and errors it encounters during analysis of
     simulation log file.}

%%%%%%%%%%%%%%%%%%%%%%%%%%%%%%%%%%%%%%%%%%%%%%%%%%%%%%%%%%%%%%%%%%%%%%%%%%%%%%%%%%%%%%%%%

\ReqItem{1}
    {Script shall allow ignoring of errors/warnings if specified in a simulation config
     file.}

%%%%%%%%%%%%%%%%%%%%%%%%%%%%%%%%%%%%%%%%%%%%%%%%%%%%%%%%%%%%%%%%%%%%%%%%%%%%%%%%%%%%%%%%%

\ReqItem{1}
    {If "verbose" option of the script is set, the script shall print the lines from the
     log file which cause the test to fail. If it is not set, the script shall only print
     the result of the test.}

%%%%%%%%%%%%%%%%%%%%%%%%%%%%%%%%%%%%%%%%%%%%%%%%%%%%%%%%%%%%%%%%%%%%%%%%%%%%%%%%%%%%%%%%%

\ReqEnd


%%%%%%%%%%%%%%%%%%%%%%%%%%%%%%%%%%%%%%%%%%%%%%%%%%%%%%%%%%%%%%%%%%%%%%%%%%%%%%%%%%%%%%%%%
% TS_SIM_REGRESS
%%%%%%%%%%%%%%%%%%%%%%%%%%%%%%%%%%%%%%%%%%%%%%%%%%%%%%%%%%%%%%%%%%%%%%%%%%%%%%%%%%%%%%%%%
\pagebreak

\section{ts_sim_regress.py}

\ReqStart{TS_SREG}

%%%%%%%%%%%%%%%%%%%%%%%%%%%%%%%%%%%%%%%%%%%%%%%%%%%%%%%%%%%%%%%%%%%%%%%%%%%%%%%%%%%%%%%%%

\ReqItem{0}
    {Scripting system shall implement \textbf{ts_sim_regress.py} script whose purpose is
     to run regressions in parallel (running multiple simulations at a time). Further in
     this section \textbf{ts_sim_regress.py} is referred to only as script. This script
     is intended to be used in CI environment.}

%%%%%%%%%%%%%%%%%%%%%%%%%%%%%%%%%%%%%%%%%%%%%%%%%%%%%%%%%%%%%%%%%%%%%%%%%%%%%%%%%%%%%%%%%

\ReqItem{0}
    {Script shall accept the same test names as \textbf{ts_sim_run.py} script.}

%%%%%%%%%%%%%%%%%%%%%%%%%%%%%%%%%%%%%%%%%%%%%%%%%%%%%%%%%%%%%%%%%%%%%%%%%%%%%%%%%%%%%%%%%

\ReqItem{0}
    {It shall be possible to use wild-cards to start multiple tests at the same time
     (e.g. to run all tests from "base_tests" group you can use "base_test/* as a test
     name)}

%%%%%%%%%%%%%%%%%%%%%%%%%%%%%%%%%%%%%%%%%%%%%%%%%%%%%%%%%%%%%%%%%%%%%%%%%%%%%%%%%%%%%%%%%

\ReqItem{0}
    {It shall be possible to specify how many simulations shall be run in parallel. When
     running simulations in parallel.}

%%%%%%%%%%%%%%%%%%%%%%%%%%%%%%%%%%%%%%%%%%%%%%%%%%%%%%%%%%%%%%%%%%%%%%%%%%%%%%%%%%%%%%%%%

\ReqItem{0}
    {When the whole regression finishes, the script shall report a summary of all tests
     indicating whether they passed or failed. The script shall also create a backup of
     all logs from regression into common directory.}

%%%%%%%%%%%%%%%%%%%%%%%%%%%%%%%%%%%%%%%%%%%%%%%%%%%%%%%%%%%%%%%%%%%%%%%%%%%%%%%%%%%%%%%%%

\ReqItem{0}
    {If there is at least one failing test in the regression, the script shall return
     non-zero exit code. This allows calling the script from CI environment, and
     connecting pipeline status to return value of this script.}
			
%%%%%%%%%%%%%%%%%%%%%%%%%%%%%%%%%%%%%%%%%%%%%%%%%%%%%%%%%%%%%%%%%%%%%%%%%%%%%%%%%%%%%%%%%

\ReqEnd


%%%%%%%%%%%%%%%%%%%%%%%%%%%%%%%%%%%%%%%%%%%%%%%%%%%%%%%%%%%%%%%%%%%%%%%%%%%%%%%%%%%%%%%%%
% TS_SIM_REQ_TRACE
%%%%%%%%%%%%%%%%%%%%%%%%%%%%%%%%%%%%%%%%%%%%%%%%%%%%%%%%%%%%%%%%%%%%%%%%%%%%%%%%%%%%%%%%%
\pagebreak

\section{ts_sim_req_trace.py}

\ReqStart{TS_RQTR}

%%%%%%%%%%%%%%%%%%%%%%%%%%%%%%%%%%%%%%%%%%%%%%%%%%%%%%%%%%%%%%%%%%%%%%%%%%%%%%%%%%%%%%%%%

\ReqItem{0}
    {Scripting system shall implement \textbf{ts_sim_req_trace.py} script whose main 
     purpose is to perform requirement tracing.}

\ReqItem{0}
    {Requirement tracing shall detect following mapping:
    \begin{item}
        \item {Test to Verification item}
        \item {Test to Design requirement}
        \item {Verification item to Design Requirement}
    \end{item}
    }

\ReqItem{0}
    {An input to requirement tracing shall be:
        \begin{itemize}
            \item{target - As defined in simulation config file}
            \item{Verification plan (written with ts-latex-lib)}
            \item{Design specification (written with ts-latex-lib)}
        \end{itemize}
    }

\ReqItem{0}
    {An output of requirement tracing shall be list with mapping from tests and
     verification items to design requirements.
    }

\ReqItem{0}
    {Output of requirement tracing shall mark design requirement as verified/not
     verified based on following criteria:
     \begin{itemize}
         \item{By default all requirements are marked as not verified.}
         \item{If a requirement is in state "Not designed", it shall never be
               marked as verified regardless of following criteria.}
         \item{If a requirement is marked as "Obsolete" it shall not be traced.}
         \item{If a requirement is linked from a test within test list file,
               it shall be marked as Verified.}
         \item{If a requirement is linked from verification item, this verification
               item is marked as "VERIFIED" and there exists a link from test to
               verification item, then requirement can be marked as "Verified".}
     \end{itemize}
    }

\ReqItem{0}
    {Requirement tracing shall output following:
     \begin{itemize}
        \item{List of Verified design requirements.}
        \item{List of Un-Verified design requirements.}
        \item{List of Un-Verified Verification items (either due to being in NOT VERIFIED state),
              or due to having no tracing from a test.}
        \item{List of Verification items having no matching design requirements.}
     \end{itemize}
    }

\ReqItem{0}
    {Requirement tracing shall indicate verification as complete if:
     \begin{itemize}
        \item{There are no Un-Verified design requirements.}
        \item{There are no Un-Verified Verification items.}
        \item{There are no Verification items which does no trace to a design requirement.}
     \end{itemize}
    }


\ReqEnd



%%%%%%%%%%%%%%%%%%%%%%%%%%%%%%%%%%%%%%%%%%%%%%%%%%%%%%%%%%%%%%%%%%%%%%%%%%%%%%%%%%%%%%%%%
% TS_SIM_COVERAGE
%%%%%%%%%%%%%%%%%%%%%%%%%%%%%%%%%%%%%%%%%%%%%%%%%%%%%%%%%%%%%%%%%%%%%%%%%%%%%%%%%%%%%%%%%
\pagebreak

\section{ts_sim_coverage.py}

\OpenIssue{Specify how should coverage analysis behave! Currently we only need to run
		   simulations. Code coverage analysis comes later during the verification
		   process, therefore we don't implement the script yet}


%%%%%%%%%%%%%%%%%%%%%%%%%%%%%%%%%%%%%%%%%%%%%%%%%%%%%%%%%%%%%%%%%%%%%%%%%%%%%%%%%%%%%%%%%
% Open issues
%%%%%%%%%%%%%%%%%%%%%%%%%%%%%%%%%%%%%%%%%%%%%%%%%%%%%%%%%%%%%%%%%%%%%%%%%%%%%%%%%%%%%%%%%
\pagebreak
\section{Open Issues}

\PrintOpenIssueSummary

\end{document}